% MOTIVATION

There is a desire for indoor positioning that can be used in different occasions, such as in indoor navigation, location-based notifications, and optimisation of work efficiency. Indoor navigation could be used for helping the user navigate through office buildings, airports, hospitals etc. This could help users save time and give them a better experience. When the location of the user is known, it is also possible to send them messages based on their location in the airport, store etc. For instance, you could send them a message about an offer on a product when you can see that they are in the duty-free shop in the airport. By using indoor location, you could also gain insight into the movement behaviour of people in a certain place. With this information, you could see if your staff always takes the shortest path, or if there are areas of your store that people do not visit. With this information, we could create a more intuitive architecture of our buildings.\cite{IPSMapsPeople}
%https://blog.mapspeople.com/mapsindoors/indoor-positioning-101

Many are familiar with \gls{gps} for navigation, but \gls{gps} is simply not accurate enough, when used indoor. Indoor positioning is therefore a field in computer science that is still very relevant, because the solutions that exist are either not accurate enough, or expensive and difficult to implement. We will therefore try to explore the different possibilities for indoor positioning in this report. We will cover the different types of sensors and actuators, and the existing positioning methods. We will use the knowledge gained in the problem analysis to choose the experiments we will conduct. We will experiment with different types of data collected on a smartphone, such as data from Wi-Fi signals, Bluetooth, and from sensors in a mobile device. 

We will experiment with state of the art algorithms in indoor positioning, and create our own hybrid between our best performing location fingerprinting algorithm and an IMU based algorithm. We will evaluate each of the methods, and choose the best performing as our final solution. 


