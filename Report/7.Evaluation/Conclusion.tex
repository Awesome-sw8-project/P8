In this project, we have worked with indoor positioning for mobile devices, where we have explored different methods to estimate accurate positions. We analysed the existing technologies in indoor positioning systems, and we explored the state-of-the-art methods and algorithms to limit the amount of experiments we would like to carry out.

In \textbf{\autoref{sec:problemstatement}}, we presented our following problem statement: 

\begin{center}
    \textbf{\textit{A service or system which can estimate the indoor locations in multi-level buildings given data collected on a mobile device.}}
\end{center}

Reflecting on the problem statement, we can conclude that we succeeded in creating a system that estimates indoor locations in multi-level buildings, given the data from shopping malls provided by Kaggle, which was collected on Android mobile devices. 

Achieving this, a number of different experiments have been made to find the optimal location fingerprinting algorithm to create a hybrid between the \acrlong{pdr} algorithm and a \gls{lfp} algorithm. We have experimented with \acrlong{ann}, \acrlong{lstm}, \acrlong{gbdt} and \acrlong{pdr}. We chose the best performing \gls{lfp} algorithm, which was the \gls{gbdt} algorithm. We have also proposed our hybrid implementations, which seeks to combine \gls{lfp} with \gls{pdr}.
Among the hybrid approaches, we have decided to go with \gls{pdr} as primary and \gls{gbdt} as support, which gave a \acrlong{mpe} of 18.59 meters.