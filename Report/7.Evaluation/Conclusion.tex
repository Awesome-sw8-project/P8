In this project we have worked with indoor positioning through mobile devices, where we have explored different methods to find the most accurate. We analysed the existing technologies in indoor positioning systems, and we explored the state of the art methods and algorithms, to limit the amount of experiments we would like to carry out. 

In \textbf{\autoref{sec:problemstatement}}, we presented our problem statement, which was the following: 

\begin{center}
    \textbf{\textit{A service or system which can estimate the indoor locations in shopping malls given data collected on a mobile device.}}
\end{center}

Reflecting on the problem statement, we can conclude that we succeeded in creating a system that estimates indoor locations, given the data from shopping malls provided by Kaggle, which was collected on android mobile devices. 

Achieving this, a number of different experiments have been made, to find the optimal location fingerprinting algorithm to create a hybrid between the \acrlong{pdr} and a machine learning algorithm. We have experimented with \acrlong{ann}, \acrlong{lstm} and \acrlong{gbdt}. We choose the best performing machine learning algorithm, which was the \gls{gbdt} algorithm. 
With the hybrid approach, it was decided to go with XX as primary and XX as support, which gave a \acrlong{rmse} on XX. 
