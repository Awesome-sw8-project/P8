%\subsection{Motivation}
%\todo{RETTTELSER HER.}
%There are many reasons why indoor positioning is worth exploring. Indoor positioning can be used for several different occasions, for instance indoor navigation, location-based notifications, and optimisation of work efficiency. 
%Indoor navigation could be used for helping the user navigate through office buildings, airports, hospitals etc. This could help users save time and give them a better experience. When the location of the user is known, it is also possible to send them messages based on their location in the airport, store etc. For instance, you could send them a message about an offer on a product, when you can see that they are in the duty-free shop in the airport. 
%By using indoor location, you could also gain insight into the movement behaviour of people in a certain place. With this information, you could for instance see if your staff always takes the shortest path, or if there are areas of your store that people do not visit. With this information, we could create a more intuitive architecture of our buildings.\cite{IPSMapsPeople}
%https://blog.mapspeople.com/mapsindoors/indoor-positioning-101

\textit{When estimating indoor locations, it is relevant to investigate the problem of accurately making these estimates and what an indoor positioning system is, which will be covered in this chapter. For this project, we have chosen to participate in a competition about predicting indoor locations held by the companies Microsoft and XYZ10 on the Kaggle platform. A dataset is provided for the purpose of determining positions, which we will be using for our project. The dataset contains measures from several different sensors and actuators over a variety of buildings and floor levels. Lastly, the different possible existing positioning methods will be covered.}