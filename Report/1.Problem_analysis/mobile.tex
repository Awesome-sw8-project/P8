\section{Mobility}
As the project which we are working with is an indoor positioning system and the motivation behind the project is to estimate and track the location of user based on data from a smartphone, we also need to look at the concept mobility. This is the case as the developed solution should be mobile. We will expand on the definition of mobility, approaches for building mobile applications/services, constraints for mobile applications/services. 

\subsection{Definition of Mobility}
Mobility is defined as the ability to move and are application or service that can be used when on the move on devices like smart phones, laptops and etc. When dealing with mobile application, the architecture types for mobile applications can broadly be divided into four, namely standalone, client-server, peer-to-peer, and hybrid. A standalone application means that no external communication is necessary for the application or service to function. In the client-server based application or service, the mobile application (client) will communicate to a third party through the HTTP protocol or an equivalent protocol. The peer-to-peer mobile systems make data communication possible by short range wireless communication (E.g. Bluetooth). The hybrid type support data communication by automatically switching between the client-server and peer-to-peer.

\subsection{Mobile Software Types}
In general, mobile software applications can be categorised into three types. These types are native software applications, web applications, hybrid software applications.\cite{mobileSoftwareTypes}

In native mobile application, the implementation will be for a specific mobile OS. Developing an application to a specific OS in general has the advantage of being optimised in terms of computation and memory, and low level mechanics like sensors on the device is more accessible. The disadvantage of this type of application is that to support multiple platforms multiple implementations of the application is necessary. 