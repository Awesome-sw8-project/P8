\section{Indoor Location \& Navigation Competition}
\label{sec:kaggleComp}
Kaggle is an online community for data scientists and machine learning enthusiasts. Kaggle offers courses, code sharing, datasets, competitions, etc. Companies can create competitions where users can compete to solve a problem formulated by different companies. Microsoft Research and XYZ10 have made a competition on Kaggle called \textbf{Indoor Location \& Navigation} and is running from January 21st 2021 to May 17th 2021. For this competition, the provided dataset fits to this project, as the idea is to work with \gls{ips}. The provided dataset does focus on \gls{ips} in multilevel buildings, meaning that besides the two usual positional estimations (X and Y), a third for the level of the building (floor) also needs estimations. The overall goal of the competition is about predicting positions of smartphones in an indoor environment with real-time sensor data in buildings like shopping malls, event centers, etc.

The reason for choosing to work with this competition is mostly due to the provided dataset. The data provided by Microsoft and XYZ10 requires explicit user permission\cite{CompetitionSite}, and it would therefore not be easy to collect this type and amount of data ourselves.

For the purpose of predicting indoor locations, it makes sense to use data from a smartphone, since this is probably the most likely device to be widely used. Additionally, the issue of estimating the precise indoor locations increase for large, multi-level buildings similar to the ones present in the dataset.
%Kaggle is a great platform for finding datasets that can help achieving data science goals. Using an already existing dataset has some advantages as it is time saving since the data collection is not needed. 
%The disadvantage of using existing datasets is the reliability and validity, since it might be unknown how the data collection process was carried out.
%But since the dataset is made for this competition and this particular purpose to optimize indoor positioning, the dataset is relevant enough to use it. Because the competition is held by Microsoft Research, which is a known company and specialises in smartphones, computers, etc., the collection of the data must be somewhat reliable.
%The datasets consist of sensor data from Android phones which have been carried around in Chinese malls. 
%The participants in the competition also have the possibility to help each other by sharing code, discuss relevant topics and team up. Therefore, we deemed this competition suitable for this project. We will be able to use some of the requirements for the competition as a structure for this project. Utilising those requirements we are able to compare our results to the other participants.

%When a team has a solution, they are supposed to estimate locations from a test dataset supplied by the organizers. The teams can then submit the predictions as a CSV file. A template for this CSV file is already provided by the organizers and contains the scenarios to predict upon. 
%The dataset contains 4 columns, namely the combined ID, floor level and X- \& Y-coordinates of the exact location. The first column containing the combined IDs can be divided into three IDs describing a site, a path and a time.
The competition hosts, furthermore, offers an approach to evaluate indoor positioning techniques using \textbf{\autoref{eq:MeanPositionError}}. To this end, they have defined a CSV file containing four columns, namely the combined site ID, floor level and X- \& Y-coordinates of the estimated location. They provide the first column, and then the idea is to estimate for the entries in this column using the dataset. This evaluation could be used as supplementary to our own evaluation of the indoor positioning method.
%For each combined ID, a prediction for floor level and location has to be estimated, which can then be written to the CSV file and submitted. The submissions are then evaluated by Kaggle based on the \textit{mean position error} which is defined as:
%For each combined ID, a prediction for floor level and location has to be estimated. The submissions are then evaluated by Kaggle using \acrlong{mpe}, which is defined as follows:

\begin{equation} \label{eq:MeanPositionError}
\text{\gls{mpe}} = \frac{1}{N} \sum_{i=1}^{N}  
                                                \left( \sqrt{( \hat{x}_i - x_i )^{2} + ( \hat{y}_i - y_i )^{2}} 
                                                + p \cdot | \hat{f}_{i} - f_i | \right) 
\end{equation}

In this evaluation, \textbf{N} denotes the number of rows in the prediction. All the denotations with a hat are the predicted variables and the corresponding denotations without the hat are the actual locations. $x$ and $y$ denote the location in the building and $f$ is the floor in the building. $p$ represents the floor penalty constant of empirically 15 meters to the ceiling\cite{MicrosoftConversation}.
%When a submission has been evaluated, the score gets a ranking on the leaderboard where the smallest scores are highest ranking. The leaderboard is divided into two different leaderboards where one is public and the other is private. The public leaderboard shows the evaluation score of approximately 15\% of the test data. The private leaderboard is hidden until the competition has ended and this leaderboard shows the evaluation score of the other 85\% of the test data.\cite{CompetitionSite}

%To interact with the competition, the participants either use the website or the Kaggle \gls{api}. The \gls{api} tool is installed using Python and \gls{pip}. To download all the data for this competition using the \gls{api}, the command \textbf{kaggle competitions download -c indoor-location-navigation} is utilized. The predictions are possible to submit utilizing the command \textbf{kaggle competitions submit indoor-location-navigation -f FILE\_NAME -m MESSAGE}, where \textbf{FILE\_NAME} is the file to be submitted and \textbf{MESSAGE} is a description of the submission.\cite{KaggleAPI} If the users do not want to use the \gls{api}, these actions are available on the website as well.