\section{Existing Systems}
In this section it will be discussed existing systems within navigation and positioning. First there will be investigated broader systems and/or companies that work with navigation and afterwards will be more specific in terms of indoor navigation. This section will contribute to what technologies that are used in practice when it comes to navigation. 


\subsubsection{\gls{gps}}
When talking about navigation, one would probably think about \gls{gps}. \gls{gps} is a U.S utilization that is used by both the military and by civilians. They use different components in order to get an accurate positioning, navigation and timing services:\cite{GPSofficial}

\begin{itemize}
    \item Space component, is operated by satellites that transmit one-way signals in order to get a current position and time. 
    \item Control component, consist of different elements: 
    \begin{itemize}
        \item Monitor Stations, tracks the GPS satellites as the pass and feed observations to the Master Control Station. 
        \item Master Control Station, computes precise locations.
        \item Ground Antennas, communication with satellites. 
    \end{itemize}
    \item User component, \gls{gps} incorporated in all mobile technology and is widely used in order to get a position of applications.
\end{itemize}

However \gls{gps} does not work well indoors, there are different problems that arise when trying to get a position when indoors. One problem if the device is not in line of sight, meaning the device does not have a clear sight to the sky, in order for the satellite to have a more accurate position. Another problem occurs when a device is indoors and can not penetrate through materials, such as glass, brick, metal and so on. Also \gls{gps} is classified as a part of \gls{uhf} which also result in another problem as indoors there is a potential sources of \gls{uhf} which can interfere with the \gls{gps} signal, for example TV antennas signals can interfere with \gls{gps} signals. \cite{GPSofficial}

\subsubsection{Google}

\subsubsection{MapsPeople} % Too broad for section.
A system/solution that can be used speficily for indoor navigation is MapsPeople's platform MapsIndoors. MapsPeople specialises in indoor navigation and has been a google partner for +10 years. Other than the platform MapsIndoors, does MapsPeople also have expertise within Google Maps. 
There is not a specific technology they use for indoor finding, depending on time, budget and overall case. Some of the technology that they mention is Bluetooth beacons, WIFI positioning, positioning via magnetic fields and via lighting. \cite{mapsppl}

\subsubsection{Nearmotion} % Copied
Nearmotion is also a company that specializes in indoor navigation and has recently developed an \gls{ar} technology that can navigated you to a specific place indoors. However their main technology is the usage of Bluetooth beacons which sends radio signals to a smartphone which also is mentioned in \ref{}.\todo{indsæt ref} \cite{nearmotion}

https://nearmotion.com/news/how-indoor-navigation-works/

\subsubsection{Steerpath} % Missing source, and beacons have already been covered enough.
Steerpath specializes in making buildings smart, meaning they transform building facilities into an interactive digital space with way finding. The way they handle the way-finding problem indoors, is, like MapsPeople, by beacons. 

\subsubsection{IndoorAtlas} % Too broad and has been covered.
With IndoorAtlas, they can make solutions for any kind of sensors that can be used to indoor navigation such as geomagnetic fields, WIFI signals, bluetooth beacons, barometric pressure and pedestrian dead reckoning, meaning the sensors in a smartphone.


\subsubsection{Partial Conclusion}
When looking at the existing systems, it is very clear that most companies use beacons in order to navigate indoors. However still at lot of the companies, such as IndoorAtlas and MapsPeople, utilize a lot of different technologies, but is dependent on the costumer's case, in terms of money, time and usage. 
%GPS
%Google (whatever they do (tror ikke der er noget her, men det kan være, ellers gik på deres Tango AR))
%MapsPeople
%Nearmotion
%Steerpath

%How are they
%What technologies do they use
%Pros and cons
