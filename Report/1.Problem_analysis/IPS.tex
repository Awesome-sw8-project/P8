\section{Indoor Positioning Systems}
\todo{RETTELSER HER.}
When we are outdoor, we rely on satellite-based positioning systems, such as \gls{gps}. The issue with \gls{gps} is that it looses its accuracy when it is used indoor, because the signals from the satellites are blocked by the walls and the roofs. This is the reason why we need other technologies when we want to estimate the indoor location. \gls{ips} is a term which covers different technologies that use mobile devices to exploit their physical location. After an \gls{ips} is developed, it is possible to make use of location-based services (LBS).

% MOTIVATION
%There are many reasons why indoor positioning is worth exploring. Indoor positioning can be used for several different occasions, for instance indoor navigation, location-based notifications, and optimisation of work efficiency. 

%Indoor navigation could be used for helping the user navigate through office buildings, airports, hospitals etc. This could help users save time and give them a better experience. When the location of the user is known, it is also possible to send them messages based on their location in the airport, store etc. For instance, you could send them a message about an offer on a product, when you can see that they are in the duty-free shop in the airport. 

%By using indoor location, you could also gain insight into the movement behaviour of people in a certain place. With this information, you could for instance see if your staff always takes the shortest path, or if there are areas of your store that people do not visit. With this information, we could create a more intuitive architecture of our buildings. \cite{IPSMapsPeople}
%https://blog.mapspeople.com/mapsindoors/indoor-positioning-101%


%https://blog.mapspeople.com/mapsindoors/indoor-positioning-101