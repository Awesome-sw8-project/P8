\section{Mobility}

As the project which we are working on is an indoor positioning system and the motivation behind the project is to estimate and track the location of a user based on data from a smartphone, we also need to look at the concept of mobility, which is the overall theme of this project module. This is the case as the developed solution should be mobile. We will expand on the definition of mobility, approaches for building mobile applications/services, constraints for mobile applications/services and location-based services.

\subsection{Definition of Mobility}
Mobility is defined as the ability to be on the move, and a mobile application is anything that can be used when being on the move, such as devices as smart phones, laptops, etc. When dealing with mobile applications, the architecture types for mobile applications can broadly be divided into four, namely standalone, client-server, peer-to-peer, and hybrid. A standalone application means that no external communication is necessary for the application or service to function. In the client-server based application or service, the mobile application (client) will communicate to a third party through the HTTP protocol or an equivalent protocol. The peer-to-peer mobile systems make data communication possible by short range wireless communication (e.g. Bluetooth). The hybrid type support data communication by automatically switching between the client-server and peer-to-peer.\cite{mallick2003mobile}

\subsection{Constraints of Mobile-Software}

When developing mobile services and applications, there are multiple factors to consider. One of the issues is battery consumption, because mobile devices rely on their battery to keep working. Because many of the mobile devices are compact in size, they are often not equipped with massive batteries with long battery life. It is therefore important to focus on developing software that is not too power consuming. Another constraint is the bandwidth on mobile connections. Wired connections are usually faster than wireless connections. The constraints with bandwidth can be improved through bulking operations, compression of information before transmitting, and caching.\cite{mobileconstraints}

%\subsection{\acrlong{lbs}}
%As defined by \cite{lbs_foundations}, an \gls{lbs} is a mobile service that makes use of a mobile network and geographic information to provide a location. In order to provide geographic information, different components are utilised depending on the environment. That is, in case the environment is indoors, the geographic information comes from external hardware, such as Wi-Fi beacons, Bluetooth beacons, \gls{rfid}, etc., as mentioned in \textbf{\autoref{sec:actuator_sensor}}. Otherwise, in outdoor environments, one would usually use \gls{gps}, Wi-Fi beacons or Bluetooth beacons. Since an \gls{lbs} is also a service, the system requires a component that handles requests, for example to an \gls{lbs} server.\cite{LBSSlides}