\section{Task Definition}
Thoughout this chapter, we will be conducting experiments with the purpose of solving the problem statement. To help in the process of conducting the experiments, we need to reify the objectives of the experiments. To this end we can define subtasks with respect to the problem statement, which we can use as goals for the experiments. The problem, which we are working on, is to identify the position in an indoor environment given data collected by a smart phone. The position is specified as a X-coordinate, Y-coordinate and floor level. To this end, we can define subtasks for identifying each of the specifications:
\begin{center}
    \textbf{Estimate the X-coordinate}
\end{center}
\vspace{-0.8cm}
\begin{center}
    \textbf{Estimate the Y-coordinate}
\end{center}
\vspace{-0.8cm}
\begin{center}
    \textbf{Estimate the floor level}
\end{center}
Even though the subtasks are defined as three separate tasks, it might be ideal to combine them when conducting an experiment. E.g. it might the case that a relationship between the X- and Y-coordinate exists or between all of the subtasks. This might lead to a model which solves more than one subtasks being superior to a model which is specialised only towards one of the subtasks. 
