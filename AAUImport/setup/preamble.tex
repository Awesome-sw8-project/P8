\documentclass{report}
\usepackage[colorinlistoftodos,prependcaption,textsize=tiny,color=green!40,linecolor=black!50]{todonotes}
%\usepackage[table,xcdraw]{xcolor}
%%%%%%%%%%%%%%%%%%%%%%%%%%%%%%%%%%%%%%%%%%%%%%%%
% Language, Encoding and Fonts
% http://en.wikibooks.org/wiki/LaTeX/Internationalization
%%%%%%%%%%%%%%%%%%%%%%%%%%%%%%%%%%%%%%%%%%%%%%%%
% Select encoding of your inputs. Depends on
% your operating system and its default input
% encoding. Typically, you should use
%   Linux  : utf8 (most modern Linux distributions)
%            latin1 
%   Windows: ansinew
%            latin1 (works in most cases)
%   Mac    : applemac
% Notice that you can manually change the input
% encoding of your files by selecting "save as"
% an select the desired input encoding. 
\usepackage[utf8]{inputenc}
\DeclareUnicodeCharacter{0176}{\^Y}


% Make latex understand and use the typographic
% rules of the language used in the document.
\usepackage[danish,english]{babel}
% Use the palatino font
\usepackage[sc]{mathpazo}
\linespread{1.05}         % Palatino needs more leading (space between lines)
% Choose the font encoding
\usepackage[T1]{fontenc}
\usepackage{float}
%%%%%%%%%%%%%%%%%%%%%%%%%%%%%%%%%%%%%%%%%%%%%%%%
% Graphics and Tables
% http://en.wikibooks.org/wiki/LaTeX/Importing_Graphics
% http://en.wikibooks.org/wiki/LaTeX/Tables
% http://en.wikibooks.org/wiki/LaTeX/Colors
%%%%%%%%%%%%%%%%%%%%%%%%%%%%%%%%%%%%%%%%%%%%%%%%
% load a colour package
\usepackage{xcolor}
\definecolor{aaublue}{RGB}{33,26,82}% dark blue
% The standard graphics inclusion package
\usepackage{graphicx}
% Set up how figure and table captions are displayed
\usepackage{caption}
\captionsetup{%
  font=footnotesize,% set font size to footnotesize
  labelfont=bf % bold label (e.g., Figure 3.2) font
}
% Make the standard latex tables look so much better
\usepackage{array,booktabs}
% Enable the use of frames around, e.g., theorems
% The framed package is used in the example environment
\usepackage{framed}


\setlength{\parindent}{0em}
\setlength{\parskip}{1em}

%%%%%%%%%%%%%%%%%%%%%%%%%%%%%%%%%%%%%%%%%%%%%%%%
% Mathematics
% http://en.wikibooks.org/wiki/LaTeX/Mathematics
%%%%%%%%%%%%%%%%%%%%%%%%%%%%%%%%%%%%%%%%%%%%%%%%
% Defines new environments such as equation,
% align and split 
\usepackage{amsmath}
% Adds new math symbols
\usepackage{amssymb}
% Use theorems in your document
% The ntheorem package is also used for the example environment
% When using thmmarks, amsmath must be an option as well. Otherwise \eqref doesn't work anymore.
\usepackage[framed,amsmath,thmmarks]{ntheorem}

%%%%%%%%%%%%%%%%%%%%%%%%%%%%%%%%%%%%%%%%%%%%%%%%
% Page Layout
% http://en.wikibooks.org/wiki/LaTeX/Page_Layout
%%%%%%%%%%%%%%%%%%%%%%%%%%%%%%%%%%%%%%%%%%%%%%%%
% Change margins, papersize, etc of the document
\usepackage[
  inner=29mm,% left margin on an odd page
  outer=29mm,% right margin on an odd page
  bmargin=40mm,
  ]{geometry}
% Modify how \chapter, \section, etc. look
% The titlesec package is very configureable
\usepackage{titlesec}
%\titleformat{\chapter}[display]{\normalfont\huge\bfseries}{\chaptertitlename\ \thechapter}{20pt}{\Huge}
%\titleformat*{\section}{\normalfont\Large\bfseries}
%\titleformat*{\subsection}{\normalfont\large\bfseries}
%\titleformat*{\subsubsection}{\normalfont\normalsize\bfseries}
\usepackage[T1]{fontenc}
\usepackage{titlesec, blindtext, color}
\definecolor{gray75}{gray}{0.75}
\newcommand{\hsp}{\hspace{20pt}}
\titleformat{\chapter}[hang]{\Huge\bfseries}{\thechapter\hsp\textcolor{gray75}{|}\hsp}{0pt}{\Huge\bfseries}


%\titleformat*{\paragraph}{\normalfont\normalsize\bfseries}
%\titleformat*{\subparagraph}{\normalfont\normalsize\bfseries}

% Clear empty pages between chapters
\let\origdoublepage\cleardoublepage
\newcommand{\clearemptydoublepage}{%
  \clearpage
  {\pagestyle{empty}\origdoublepage}%
}
\let\cleardoublepage\clearemptydoublepage

% Change the headers and footers
\usepackage{fancyhdr}
\pagestyle{fancy}
\fancyhf{} %delete everything
\renewcommand{\headrulewidth}{0pt} %remove the horizontal line in the header
%\fancyhead[RE]{\small\nouppercase\leftmark} %even page - chapter title
%\fancyhead[LO]{\small\nouppercase\rightmark} %uneven page - section title
\fancyfoot[C]{\thepage} %page number on all pages
% Do not stretch the content of a page. Instead,
% insert white space at the bottom of the page
\raggedbottom
% Enable arithmetics with length. Useful when
% typesetting the layout.
\usepackage{calc}

%%%%%%%%%%%%%%%%%%%%%%%%%%%%%%%%%%%%%%%%%%%%%%%%
% Bibliography
% http://en.wikibooks.org/wiki/LaTeX/Bibliography_Management
%%%%%%%%%%%%%%%%%%%%%%%%%%%%%%%%%%%%%%%%%%%%%%%%
\usepackage[sorting=none]{biblatex}
\addbibresource{AAUImport/bib/mybib.bib}

%%%%%%%%%%%%%%%%%%%%%%%%%%%%%%%%%%%%%%%%%%%%%%%%
% Misc
%%%%%%%%%%%%%%%%%%%%%%%%%%%%%%%%%%%%%%%%%%%%%%%%
% Add bibliography and index to the table of
% contents
\usepackage[nottoc]{tocbibind}
% Add the command \pageref{LastPage} which refers to the
% page number of the last page
\usepackage{lastpage}

\usepackage{tabularx}
\usepackage{rotating}
\usepackage{tikz}
% Add todo notes in the margin of the document
  
\usepackage{minted}
\colorlet{shadecolor}{gray!15}
\newcommand{\tabitem}{~~\llap{\textbullet}~~}
\usepackage{csquotes}
\usepackage{xcolor}
\usepackage[toc,page]{appendix}
\usepackage{amssymb}
\usepackage{diagbox}

%%%%%%%%%%%%%%%%%%%%%%%%%%%%%%%%%%%%%%%%%%%%%%%%
% Hyperlinks
% http://en.wikibooks.org/wiki/LaTeX/Hyperlinks
%%%%%%%%%%%%%%%%%%%%%%%%%%%%%%%%%%%%%%%%%%%%%%%%
% Enable hyperlinks and insert info into the pdf
% file. Hypperref should be loaded as one of the 
% last packages
\usepackage{hyperref}
%\usepackage[notocbib]{apacite}
\hypersetup{ 
	%pdfpagelabels=true,%
	plainpages=false,%
	pdfauthor={Author(s)},%
	pdftitle={Title},%
	pdfsubject={Subject},%
	bookmarksnumbered=true,%
	colorlinks=false,%
	citecolor=black,%
	filecolor=black,%
	linkcolor=black,% you should probably change this to black before printing
	urlcolor=black,%
	pdfstartview=FitH%
}


\usepackage{wrapfig}
\usepackage{afterpage}
\usepackage{lscape}


% Fixme notes. http://madsn.net/index.php?title=FiXme
\usepackage[footnote,draft,danish,silent,nomargin]{fixme}


\usepackage[smartEllipses]{markdown}

%--------------- FOR TIP AND AVOID BOXES --------------------
\usepackage[framemethod=default]{mdframed}

%--------------- FOR TIP BOX --------------------
\newenvironment{tip}{\begin{erBox}}{\hfill{\tiny}\end{erBox}}

\definecolor{wrongexample}{RGB}{238,41,18}

\newmdenv[skipabove=7pt,
skipbelow=7pt,
rightline=false,
leftline=true,
topline=false,
bottomline=false,
backgroundcolor=rightexample!10,
linecolor=rightexample,
innerleftmargin=5pt,
innerrightmargin=5pt,
innertopmargin=5pt,
innerbottommargin=5pt,
leftmargin=0cm,
rightmargin=0cm,
linewidth=4pt]{erBox}
%--------------- END TIP BOX --------------------


%--------------- FOR AVOID BOX --------------------
\newenvironment{avoid}{\begin{ewBox}}{\hfill{\tiny}\end{ewBox}}

\definecolor{rightexample}{RGB}{31,181,0}

\newmdenv[skipabove=7pt,
skipbelow=7pt,
rightline=false,
leftline=true,
topline=false,
bottomline=false,
backgroundcolor=wrongexample!10,
linecolor=wrongexample,
innerleftmargin=5pt,
innerrightmargin=5pt,
innertopmargin=5pt,
innerbottommargin=5pt,
leftmargin=0cm,
rightmargin=0cm,
linewidth=4pt]{ewBox}
%--------------- END AVOID BOX --------------------

\newenvironment{issuebox}{\begin{eiBox}}{\hfill{\tiny}\end{eiBox}}

\definecolor{issue}{RGB}{0,142,204}
%\definecolor{issueexample}{RGB}{31,181,0}

\newmdenv[skipabove=7pt,
skipbelow=7pt,
rightline=false,
leftline=true,
topline=false,
bottomline=false,
backgroundcolor=issue!10,
linecolor=issue,
innerleftmargin=5pt,
innerrightmargin=5pt,
innertopmargin=5pt,
innerbottommargin=5pt,
leftmargin=0cm,
rightmargin=0cm,
linewidth=4pt]{eiBox}


\newenvironment{goal}{\begin{egBox}}{\hfill{\tiny}\end{egBox}}

\definecolor{goalcolor}{rgb}{0.44, 0.16, 0.39}
%\definecolor{issueexample}{RGB}{31,181,0}


\newmdenv[skipabove=7pt,
skipbelow=7pt,
rightline=false,
leftline=true,
topline=false,
bottomline=false,
backgroundcolor=goalcolor!10,
linecolor=goalcolor,
innerleftmargin=5pt,
innerrightmargin=5pt,
innertopmargin=5pt,
innerbottommargin=5pt,
leftmargin=0cm,
rightmargin=0cm,
linewidth=4pt]{egBox}

\usepackage{figsize}


% ¤¤ Opsaetning af listings ¤¤ %
\definecolor{commentGreen}{RGB}{34,139,24}
\definecolor{stringPurple}{RGB}{208,76,239}

% ¤¤ Misc. ¤¤ %
\usepackage{listings}						% Placer kildekode i dokumentet med \begin{lstlisting}...\end{lstlisting}

\usepackage{xcolor} 
% farver til syntax highlighting i vores sprog

\usepackage{euscript}

\definecolor{brickred}{rgb}{0.8, 0.25, 0.33}
\definecolor{brightgreen}{rgb}{0.2539, 0.793, 0.3633}
\definecolor{corn}{rgb}{0.98, 0.93, 0.36}


\newcommand\codeil{\lstinline}
\newcommand\codeilblack{\lstinline[keywordstyle={}]}

\newcommand*{\todoerror}[1]{\todo[color=brickred]{#1}}
\newcommand*{\todowarning}[1]{\todo[color=corn]{#1}}
\newcommand*{\todomessage}[1]{\todo[color=brightgreen]{#1}}

\usepackage[ruled,vlined,linesnumbered]{algorithm2e}
\usepackage{adjustbox}

