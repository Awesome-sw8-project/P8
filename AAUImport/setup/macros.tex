% see, e.g., http://en.wikibooks.org/wiki/LaTeX/Customizing_LaTeX#New_commands
% for more information on how to create macros

%%%%%%%%%%%%%%%%%%%%%%%%%%%%%%%%%%%%%%%%%%%%%%%%
% Macros for the titlepage
%%%%%%%%%%%%%%%%%%%%%%%%%%%%%%%%%%%%%%%%%%%%%%%%
%Creates the aau titlepage
\newcommand{\aautitlepage}[3]{%
  {
    %set up various length
    \ifx\titlepageleftcolumnwidth\undefined
      \newlength{\titlepageleftcolumnwidth}
      \newlength{\titlepagerightcolumnwidth}
    \fi
    \setlength{\titlepageleftcolumnwidth}{0.4\textwidth-\tabcolsep}
    \setlength{\titlepagerightcolumnwidth}{\textwidth-2\tabcolsep-\titlepageleftcolumnwidth}
    %create title page
    \thispagestyle{empty}
    \noindent%
    \begin{tabular}{@{}ll@{}}
      \parbox{\titlepageleftcolumnwidth}{
        \iflanguage{danish}{%
          \includegraphics[width=\titlepageleftcolumnwidth]{"AAUImport/figures/aau_logo_da"}
        }{%
          \includegraphics[width=125pt]{"AAUImport/figures/aau_logo_en"}
        }
      } &
      \parbox{\titlepagerightcolumnwidth}{\raggedleft\sf\small
        #2
      }\bigskip\\
       #1 &
      \parbox[t]{\titlepagerightcolumnwidth}{%
      \textbf{Abstract:}\bigskip\par
        \fbox{\parbox{\titlepagerightcolumnwidth-2\fboxsep-2\fboxrule}{%
          #3
        }}
      }\\
    \end{tabular}
    \vfill
    \iflanguage{danish}{%
      \noindent{\footnotesize\emph{Rapportens indhold er frit tilgængeligt, men offentliggørelse (med kildeangivelse) må kun ske efter aftale med forfatterne.}}
    }{%
      \noindent{\footnotesize\emph{The content of this report is freely available, but publication (with reference) may only be pursued due to agreement with the author.}}
    }
    \clearpage
  }
}

%Create english project info
\newcommand{\englishprojectinfo}[8]{%
  \parbox[t]{\titlepageleftcolumnwidth}{
    \textbf{Title:}\\ #1\bigskip\par
    \textbf{Theme:}\\ #2\bigskip\par
    \textbf{Project Period:}\\ #3\bigskip\par
    \textbf{Project Group:}\\ #4\bigskip\par
    \textbf{Participants:}\\ #5\bigskip\par
    \textbf{Supervisor:}\\ #6\bigskip\par
    %\textbf{Copies:} #7\bigskip\par
    \textbf{Page Numbers:} \pageref{LastPage}\bigskip\par
    \textbf{Date of Completion:}\\ #8
  }
}

%Create danish project info
\newcommand{\danishprojectinfo}[8]{%
  \parbox[t]{\titlepageleftcolumnwidth}{
    \textbf{Titel:}\\ #1\bigskip\par
    \textbf{Tema:}\\ #2\bigskip\par
    \textbf{Projektperiode:}\\ #3\bigskip\par
    \textbf{Projektgruppe:}\\ #4\bigskip\par
    \textbf{Deltager(e):}\\ #5\bigskip\par
    \textbf{Vejleder(e):}\\ #6\bigskip\par
    \textbf{Oplagstal:} #7\bigskip\par
    \textbf{Sidetal:} \pageref{LastPage}\bigskip\par
    \textbf{Afleveringsdato:}\\ #8
  }
}

%%%%%%%%%%%%%%%%%%%%%%%%%%%%%%%%%%%%%%%%%%%%%%%%
% An example environment
%%%%%%%%%%%%%%%%%%%%%%%%%%%%%%%%%%%%%%%%%%%%%%%%
\theoremheaderfont{\normalfont\bfseries}
\theorembodyfont{\normalfont}
\theoremstyle{break}
\def\theoremframecommand{{\color{gray!50}\vrule width 5pt \hspace{5pt}}}
\newshadedtheorem{exa}{Example}[chapter]
\newenvironment{example}[1]{%
		\begin{exa}[#1]
}{%
		\end{exa}
}

\newcounter{defcounter}
\setcounter{defcounter}{0}
\newenvironment{hypothesis}{%
\addtocounter{equation}{-1}
\refstepcounter{defcounter}
\renewcommand\theequation{H.\thedefcounter}
\begin{equation}}
{\end{equation}}

\newcounter{subcounter}
\setcounter{subcounter}{0} % Problemet er hvis man f.eks skal lave en ny subhypotese til H7, så vil den første være H.7.3 nu, for. Jo hvis det er casen, så er det no stress
%Ja, men tror du ikke det kun er hyp 5 der får subhyps?
%Ellers laver vi bare en subhypothesis2 command med en ny counter <--- trueeeee
%Nice fix *highfive*
%Nice *fistbump*
%Hvorfor fistbumper du en highfive din retard?
%Shit det var akavet, nu tør jeg ikke tage i grupperummet igen :S
%Nej du må hellere blive væk for evigt
% *græder*
%*griner*
%*grinder*? - nej
%Og så levede de lykkeligt til deres dages ende <- det her

\newenvironment{subhypothesis}{%
%\addtocounter{equation}{0}
\refstepcounter{subcounter} %denne linie incrementer subcounter, tror jeg
\renewcommand\theequation{H.5.\thesubcounter}
\begin{equation}}
{\end{equation}}

% \newcounter{snippetcounter}
% \setcounter{snippetcounter}{0}
% \newenvironment{pythonsnippet}{%
% \addtocounter{figure}{-1}
% \refstepcounter{snippetcounter}
% \renewcommand\thefigure{Code Snippet \thesnippetcounter}
% \begin{figure}[H]
% \begin{cmintedlinenos}[linenos]{python}}
% {\end{cmintedlinenos}\end{figure}}