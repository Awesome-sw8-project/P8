\pdfbookmark[0]{English title page}{label:titlepage_en}
\aautitlepage{%
  \englishprojectinfo{
    \textit{Boats and Hoes}%title
  }{%
    8th Semester Project (Mobility) %theme
  }{%
    Fall Semester 2021 %project period
  }{%
    SW807F21
  }{%
    %list of group members
    Abiram Mohanaraj, \\
    Cecilie Hyrup Madsen,\\
    Elisabeth Niemeyer Laursen, \\
    Martin Pekár Christensen, \\
    Melanie Selman,\\
    Mikkel Filip Jensen\\
  }{%
    %list of supervisors
    Supervisor
  }{%
    1 % number of printed copies
  }{%
    26th of May, 2021 % date of completion
  }%
}{%department and address
  \textbf{Department of Computer Science}\\
  Aalborg University\\
  \href{http://www.aau.dk}{http://www.aau.dk}
}{
%\todo{Husk ikke at bruge passive voice. Derudover er projektet ikke basered rundt om Kaggle. Vi bruger bare deres dataset.}
This project investigates the difficulties working with indoor navigation as positioning people indoors can be a complex task as there for now is no optimal solution. Therefore, the goal of this project is to predict indoor positions based on real-time sensor data. 
To meet this goal, several experiments are made to find the most accurate solution. The experiments consist of multiple methods, such as machine learning, Inertial Measurement Unit (IMU) methods, and lastly hybrid methods. To train these models, a dataset provided by Kaggle is used, which consists of multiple indoor positions with matching sensor readings.
\\
As result, we went with a hybrid approach, combining the IMU method Pedestrian Dead Reckoning (PDR) with the best performing machine learning model Gradient Boosting Decision Tree (GBDT) with a Mean Position Error on 23.21. 
Different compositions has been experimented with and is concluded that PDR as primary and the GBDT as support is most sufficient, giving the Mean Positioning Error on 18.59.   
}
%\cleardoublepagehttps://www.overleaf.com/project/5f51e9d829099c000199193c
%\afterpage{\blankpage}