\pdfbookmark[0]{English title page}{label:titlepage_en}
\aautitlepage{%
  \englishprojectinfo{
    \textit{Indoor Position Estimation in Multi-Level Buildings}%title
  }{%
    8th Semester Project (Mobility) %theme
  }{%
    Spring Semester 2021 %project period
  }{%
    SW807F21
  }{%
    %list of group members
    Abiram Mohanaraj, \\
    Cecilie Hyrup Madsen,\\
    Elisabeth Niemeyer Laursen, \\
    Martin Pekár Christensen, \\
    Melanie Selman,\\
    Mikkel Filip Jensen\\
  }{%
    %list of supervisors
    Dalin Zhang
  }{%
    1 % number of printed copies
  }{%
    27th of May, 2021 % date of completion
  }%
}{%department and address
  \textbf{Department of Computer Science}\\
  Aalborg University\\
  \href{http://www.aau.dk}{http://www.aau.dk}
}{
%\todo{Husk ikke at bruge passive voice. Derudover er projektet ikke basered rundt om Kaggle. Vi bruger bare deres dataset.}
In this project, we investigate the task of indoor positioning in multi-level buildings. This can be a complex task as there currently are no optimal solutions. Therefore, the goal of this project is to predict indoor positions based on real-time sensor data from smartphones. 
To meet this goal, we conduct several experiments to find the most accurate solution. We also propose three hybrid architectures, which incorporates Pedestrian Dead Reckoning (PDR) and Location Fingerprinting (LFP). Our experiments consist of multiple algorithms within LFP and IMU based methods, and lastly our proposed hybrid methods. To evaluate these methods, we use a dataset provided by the Indoor Navigation \& Location competition at Kaggle, which consists of a variety of sensor data from smartphones and location data.
\\

We also evaluate the positioning methods by comparing the Mean Position Error (MPE). We also conclude that a hybrid approach, combining the IMU method Pedestrian Dead Reckoning (PDR) as primary with Gradient Boosting Decision Tree (GBDT) as support yields the best performance with a Mean Position Error (MPE) of 18.59. GBDT was chosen since it performed best with a MPE of 23.21. The performance of our best hybrid solution performed 19.91\% better compared to GBDT.
}
%\cleardoublepagehttps://www.overleaf.com/project/5f51e9d829099c000199193c
%\afterpage{\blankpage}